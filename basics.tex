\section{Chapter 1: Basics}

\subsection{What is a Category?}
\begin{definition}
  A \textbf{category} consists of:
  \begin{itemize}
    \item  a collection of \textbf{objects} ($X, Y, Z$, etc.)
    \item a collection of \textbf{morphisms} ($f, g, h$, etc.) where the morphisms have \textit{specified} \textbf{domain} and \textbf{codomain} objects, i.e. $f : X \to Y$
  \end{itemize}
  such that
  \begin{itemize}
    \item each object has an \textbf{identity} morphism $\id_X : X \to X$
    \item \textbf{composition} of morphisms gives a new morphism, so that if $f : X \to Y$, and $g: Y \to Z$, then $g \circ f = g f : X \to Z$.
    \item the identity behaves as expected with respect to composition, i.e. for all $f : X \to Y$, that $f \id_X = \id_Y f = f$.
    \item the composition of morphisms is \textbf{associative}, so if we have $f : X \to Y$, $g : Y \to Z$, and $h : Z \to W$, then $(hg)f = h(gf) := hgf : X \to W$.
  \end{itemize}
\end{definition}

\begin{example}
  List of examples of some common categories:
  \begin{itemize}
    \item $\cat{Set}$, the category with sets as objects and regular functions as morphisms
    \item $\cat{Grp}$, the category with groups as objects and group homomorphisms as morphisms
    \item $\cat{Top}$, the category with topological spaces as objects and continuous maps as morphisms
    \item
  \end{itemize}
  Note that categories are often named after their objects.
\end{example}

\subsection{Opposite Day}
\begin{definition}
  If $\cat C$ is a category, we define the \textbf{opposite category} $\opp{\cat C}$
\end{definition}


\subsection{What is a Functor?}

\subsection{What is a Natural Transformation?}

\subsection{Equivalence of Categories}

\subsection{(there might be more important stuff here that I skipped idk)}